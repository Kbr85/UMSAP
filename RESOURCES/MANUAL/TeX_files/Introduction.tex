\chapter{Introduction}

Utilities for Mass Spectrometry Analysis of Proteins (UMSAP) is a graphical user
interface (GUI) designed to speed up the post-processing of data obtained during
mass spectrometry studies involving proteins. The program is not intended to
analyze a mass spectrum or a mass chromatogram, neither to identify the peaks
in a mass spectrum. The main objective is the fast post-processing of the vast
amount of data generated in mass spectrometry experiments involving proteins
after peak identification have been performed.

The program is organized in modules with each module performing a single type
of data post-processing. The reason for this clear separation is the high
dependency between the type of mass spectrometry experiment performed and the
way in which the resulting data must be post-processed. The modules are designed
in such a way that the required user input is minimized but still users can
control every aspect of the analysis. Currently, the software contains three
modules, but several others are already planned. 

\section{Citing Utilities for Mass Spectrometry Analysis of Proteins}

If results obtained with UMSAP are published in any way, please acknowledge the
use of UMSAP by including the following sentence:

"Utilities for Mass Spectrometry Analysis of Proteins was created by Kenny Bravo
Rodriguez at the University of Duisburg-Essen and is currently developed at the
Max Planck Institute of Molecular Physiology."

Any published work, which uses UMSAP, should include the following reference:

Kenny Bravo-Rodriguez, Birte Hagemeier, Lea Drescher, Marian Lorenz,
Michael Meltzer, Farnusch Kaschani, Markus Kaiser and Michael Ehrmann.
(\num{2018}). Utilities for Mass Spectrometry Analysis of Proteins (UMSAP): Fast
post-processing of mass spectrometry data.
\href{https://onlinelibrary.wiley.com/doi/10.1002/rcm.8243}{Rapid Communications
in Mass Spectrometry}, \num{32}(\num{19}),
\numrange[range-phrase = --]{1659}{1667}.

\newpage

Electronic documents should include a direct link to the official web page of
UMSAP at: \href{https://www.umsap.nl}{www.umsap.nl}

\section{Acknowledgments}

I would like to thank all the persons that have contributed to the development
of UMSAP, either by contributing ideas and suggestions or by testing the code.
Special thanks go to: Dr. Farnusch Kaschani, Dr. Juliana Rey, Dr. Petra Janning
and Prof. Dr. Daniel Hoffmann.

In particular, I would like to thank Prof. Dr. Michael Ehrmann.

