\chapter{Legal details}
\label{chap:licagre}

\section{License Agreement}

Utilities for Mass Spectrometry Analysis of Proteins and its source code are governed by the following license:

Upon execution of this Agreement by the party identified below ("Licensee"), Kenny Bravo Rodriguez (KBR) will provide the Utilities for Mass Spectrometry Analysis of Proteins software in Executable Code and/or Source Code form ("Software") to Licensee, subject to the following terms and conditions. For purposes of this Agreement, Executable Code is the compiled code, which is ready to run on Licensee's computer. Source code consists of a set of files, which contain the actual program commands that are compiled to form the Executable Code.

1. The Software is intellectual property owned by KBR, and all rights, title and interest, including copyright, remain with KBR. KBR grants, and Licensee hereby accepts, a restricted, non-exclusive, non-transferable license to use the Software for academic, research and internal business purposes only, e.g. not for commercial use (see Clause 7 below), without a fee.

2. Licensee may, at its own expense, create and freely distribute complimentary works that inter-operate with the Software, directing others to the Utilities for Mass Spectrometry Analysis of Proteins web page to license and obtain the Software itself. Licensee may, at its own expense, modify the Software to make derivative works. Except as explicitly provided below, this License shall apply to any derivative work as it does to the original Software distributed by KBR. Any derivative work should be clearly marked and renamed to notify users that it is a modified version and not the original Software distributed by KBR. Licensee agrees to reproduce the copyright notice and other proprietary markings on any derivative work and to include in the documentation of such work the acknowledgment: "This software includes code developed by Kenny Bravo Rodriguez for the Utilities for Mass Spectrometry Analysis of Proteins software".

Licensee may not sell any derivative work based on the Software under any circumstance. For commercial distribution of the Software or any derivative work based on the Software a separate license is required. Licensee may contact KBR to negotiate an appropriate license for such distribution.

3. Except as expressly set forth in this Agreement, THIS SOFTWARE IS PROVIDED "AS IS" AND KBR MAKES NO REPRESENTATIONS AND EXTENDS NO WARRANTIES OF ANY KIND, EITHER EXPRESS OR IMPLIED, INCLUDING BUT NOT LIMITED TO WARRANTIES OR MERCHANTABILITY OR FITNESS FOR A PARTICULAR PURPOSE, OR THAT THE USE OF THE SOFTWARE WILL NOT INFRINGE ANY PATENT, TRADEMARK, OR OTHER RIGHTS. LICENSEE ASSUMES THE ENTIRE RISK AS TO THE RESULTS AND PERFORMANCE OF THE SOFTWARE AND/OR ASSOCIATED MATERIALS. LICENSEE AGREES THAT KBR SHALL NOT BE HELD LIABLE FOR ANY DIRECT, INDIRECT, CONSEQUENTIAL, OR INCIDENTAL DAMAGES WITH RESPECT TO ANY CLAIM BY LICENSEE OR ANY THIRD PARTY ON ACCOUNT OF OR ARISING FROM THIS AGREEMENT OR USE OF THE SOFTWARE AND/OR ASSOCIATED MATERIALS.

4. Licensee understands the Software is proprietary to KBR. Licensee agrees to take all reasonable steps to insure that the Software is protected and secured from unauthorized disclosure, use, or release and will treat it with at least the same level of care as Licensee would use to protect and secure its own proprietary computer programs and/or information, but using no less than a reasonable standard of care.  Licensee agrees to provide the Software only to any other person or entity who has registered with KBR. If Licensee is not registering as an individual but as an institution or corporation each member of the institution or corporation who has access to or uses Software must agree to and abide by the terms of this license. If Licensee becomes aware of any unauthorized licensing, copying or use of the Software, Licensee shall promptly notify KBR in writing. Licensee expressly agrees to use the Software only in the manner and for the specific uses authorized in this Agreement.

5. KBR shall have the right to terminate this license immediately by written notice upon Licensee's breach of, or non-compliance with, any terms of the license. Licensee may be held legally responsible for any copyright infringement that is caused or encouraged by its failure to abide by the terms of this license. Upon termination, Licensee agrees to destroy all copies of the Software in its possession and to verify such destruction in writing.

6. Licensee agrees that any reports or published results obtained with the Software will acknowledge its use by the appropriate citation as follows:

"Utilities for Mass Spectrometry Analysis of Proteins was developed by Kenny Bravo Rodriguez at the University of Duisburg-Essen."

Any published work, which utilizes Utilities for Mass Spectrometry Analysis of Proteins, shall include the following reference:

Kenny Bravo-Rodriguez, Birte Hagemeier, Lea Drescher, Marian Lorenz, Michael Meltzer, Farnusch Kaschani, Markus Kaiser and Michael Ehrmann. (\num{2018}). Utilities for Mass Spectrometry Analysis of Proteins (UMSAP): Fast post-processing of mass spectrometry data. \href{https://onlinelibrary.wiley.com/doi/10.1002/rcm.8243}{Rapid Communications in Mass Spectrometry}, \num{32}(\num{19}), \numrange[range-phrase = --]{1659}{1667}.

Electronic documents will include a direct link to the official Utilities for Mass Spectrometry Analysis of Proteins page at:
\href{https://www.umsap.nl/}{www.umsap.nl}

7. Commercial use of the Software, or derivative works based thereon, REQUIRES A COMMERCIAL LICENSE.  Should Licensee wish to make commercial use of the Software, Licensee will contact KBR to negotiate an appropriate license for such use. Commercial use includes: 
(1) integration of all or part of the Software into a product for sale, lease or license by or on behalf of Licensee to third parties, or 
(2) distribution of the Software to third parties that need it to commercialize product sold or licensed by or on behalf of Licensee.

8. Utilities for Mass Spectrometry Analysis of Proteins is being distributed as a research tool and as such, KBR encourages contributions from users of the code that might, at KBR's sole discretion, be used or incorporated to make the basic operating framework of the Software a more stable, flexible, and/or useful product. Licensees who contribute their code to become an internal portion of the Software agree that such code may be distributed by KBR under the terms of this License and may be required to sign an "Agreement Regarding Contributory Code for Utilities for Mass Spectrometry Analysis of Proteins Software" before KBR can accept it (contact umsap-licenses@umsap.nl for a copy).

UNDERSTOOD AND AGREED.

Contact Information:

The best contact path for licensing issues is by e-mail to umsap-licenses@umsap.nl

\section{Copyrights Notes}

UMSAP \num[parse-numbers=false]{2.1.0} is written in Python and uses the following modules and Python version:
\begin{table}[h!]
	\centering
	\begin{tabular}{l c}
		\hline
		Module & Version \\
		\hline
		Biopython   & \num{1.73} \\
		Matplotlib  & \num[parse-numbers=false]{3.0.2} \\
		NumPy       & \num[parse-numbers=false]{1.16.1}\\
		PyInstaller & \num{3.4}\\
		Python      & \num[parse-numbers=false]{3.7.1}\\
		Requests    & \num[parse-numbers=false]{2.21.0}\\
		wxPython    & \num[parse-numbers=false]{4.0.4}\\
		\hline		
	\end{tabular}
	\caption[List of modules used by UMSAP]{\textbf{List of modules used by UMSAP.}}
	\label{table:umsappythonmodules}
\end{table}

The Copyrights Notes or License Agreements for the modules are as follow:

\textbf{Biopython}

Biopython is currently released under the "Biopython License Agreement" (given in full below). Unless stated otherwise in individual file headers, all Biopython's files are under the "Biopython License Agreement".

Some files are explicitly dual licensed under your choice of the "Biopython License Agreement" or the "BSD 3-Clause License" (both given in full below). This is with the intention of later offering all of Biopython under this dual licensing approach.

Biopython License Agreement

Permission to use, copy, modify, and distribute this software and its documentation with or without modifications and for any purpose and without fee is hereby granted, provided that any copyright notices appear in all copies and that both those copyright notices and this permission notice appear in supporting documentation, and that the names of the contributors or copyright holders not be used in advertising or publicity pertaining to distribution of the software without specific prior permission.

THE CONTRIBUTORS AND COPYRIGHT HOLDERS OF THIS SOFTWARE DISCLAIM ALL WARRANTIES WITH REGARD TO THIS SOFTWARE, INCLUDING ALL IMPLIED WARRANTIES OF MERCHANTABILITY AND FITNESS, IN NO EVENT SHALL THE CONTRIBUTORS OR COPYRIGHT HOLDERS BE LIABLE FOR ANY SPECIAL, INDIRECT OR CONSEQUENTIAL DAMAGES OR ANY DAMAGES WHATSOEVER RESULTING FROM LOSS OF USE, DATA OR PROFITS, WHETHER IN AN ACTION OF CONTRACT, NEGLIGENCE OR OTHER TORTIOUS ACTION, ARISING OUT OF OR IN CONNECTION WITH THE USE OR PERFORMANCE OF THIS SOFTWARE.

BSD 3-Clause License

Copyright (c) 1999-2019, The Biopython Contributors All rights reserved.

Redistribution and use in source and binary forms, with or without modification, are permitted provided that the following conditions are met:

Redistributions of source code must retain the above copyright notice, this list of conditions and the following disclaimer.

Redistributions in binary form must reproduce the above copyright notice, this list of conditions and the following disclaimer in the documentation and/or other materials provided with the distribution.
Neither the name of the copyright holder nor the names of its contributors may be used to endorse or promote products derived from this software without specific prior written permission.

THIS SOFTWARE IS PROVIDED BY THE COPYRIGHT HOLDERS AND CONTRIBUTORS "AS IS" AND ANY EXPRESS OR IMPLIED WARRANTIES, INCLUDING, BUT NOT LIMITED TO, THE IMPLIED WARRANTIES OF MERCHANTABILITY AND FITNESS FOR A PARTICULAR PURPOSE ARE DISCLAIMED. IN NO EVENT SHALL THE COPYRIGHT HOLDER OR CONTRIBUTORS BE LIABLE FOR ANY DIRECT, INDIRECT, INCIDENTAL, SPECIAL, EXEMPLARY, OR CONSEQUENTIAL DAMAGES (INCLUDING, BUT NOT LIMITED TO, PROCUREMENT OF SUBSTITUTE GOODS OR SERVICES; LOSS OF USE, DATA, OR PROFITS; OR BUSINESS INTERRUPTION) HOWEVER CAUSED AND ON ANY THEORY OF LIABILITY, WHETHER IN CONTRACT, STRICT LIABILITY, OR TORT (INCLUDING NEGLIGENCE OR OTHERWISE) ARISING IN ANY WAY OUT OF THE USE OF THIS SOFTWARE, EVEN IF ADVISED OF THE POSSIBILITY OF SUCH DAMAGE.

\textbf{Matplotlib}

Copyright Policy

John Hunter began matplotlib around 2003. Since shortly before his passing in 2012, Michael Droettboom has been the lead maintainer of matplotlib, but, as has always been the case, matplotlib is the work of many.

Prior to July of 2013, and the 1.3.0 release, the copyright of the source code was held by John Hunter. As of July 2013, and the 1.3.0 release, matplotlib has moved to a shared copyright model.

matplotlib uses a shared copyright model. Each contributor maintains copyright over their contributions to matplotlib. But, it is important to note that these contributions are typically only changes to the repositories. Thus, the matplotlib source code, in its entirety, is not the copyright of any single person or institution. Instead, it is the collective copyright of the entire matplotlib Development Team. If individual contributors want to maintain a record of what changes/contributions they have specific copyright on, they should indicate their copyright in the commit message of the change, when they commit the change to one of the matplotlib repositories.

The Matplotlib Development Team is the set of all contributors to the matplotlib project. A full list can be obtained from the git version control logs.
License agreement for matplotlib 3.0.3

1. This LICENSE AGREEMENT is between the Matplotlib Development Team ("MDT"), and the Individual or Organization ("Licensee") accessing and otherwise using matplotlib software in source or binary form and its associated documentation.

2. Subject to the terms and conditions of this License Agreement, MDT hereby grants Licensee a nonexclusive, royalty-free, world-wide license to reproduce, analyze, test, perform and/or display publicly, prepare derivative works, distribute, and otherwise use matplotlib 3.0.3 alone or in any derivative version, provided, however, that MDT's License Agreement and MDT's notice of copyright, i.e., "Copyright (c) 2012-2013 Matplotlib Development Team; All Rights Reserved" are retained in matplotlib 3.0.3 alone or in any derivative version prepared by Licensee.

3. In the event Licensee prepares a derivative work that is based on or incorporates matplotlib 3.0.3 or any part thereof, and wants to make the derivative work available to others as provided herein, then Licensee hereby agrees to include in any such work a brief summary of the changes made to matplotlib 3.0.3.

4. MDT is making matplotlib 3.0.3 available to Licensee on an "AS IS" basis. MDT MAKES NO REPRESENTATIONS OR WARRANTIES, EXPRESS OR IMPLIED. BY WAY OF EXAMPLE, BUT NOT LIMITATION, MDT MAKES NO AND DISCLAIMS ANY REPRESENTATION OR WARRANTY OF MERCHANTABILITY OR FITNESS FOR ANY PARTICULAR PURPOSE OR THAT THE USE OF MATPLOTLIB 3.0.3 WILL NOT INFRINGE ANY THIRD PARTY RIGHTS.

5. MDT SHALL NOT BE LIABLE TO LICENSEE OR ANY OTHER USERS OF MATPLOTLIB 3.0.3 FOR ANY INCIDENTAL, SPECIAL, OR CONSEQUENTIAL DAMAGES OR LOSS AS A RESULT OF MODIFYING, DISTRIBUTING, OR OTHERWISE USING MATPLOTLIB 3.0.3, OR ANY DERIVATIVE THEREOF, EVEN IF ADVISED OF THE POSSIBILITY THEREOF.

6. This License Agreement will automatically terminate upon a material breach of its terms and conditions.

7. Nothing in this License Agreement shall be deemed to create any relationship of agency, partnership, or joint venture between MDT and Licensee. This License Agreement does not grant permission to use MDT trademarks or trade name in a trademark sense to endorse or promote products or services of Licensee, or any third party.

8. By copying, installing or otherwise using matplotlib 3.0.3, Licensee agrees to be bound by the terms and conditions of this License Agreement.
\newpage

\textbf{NumPy}

Copyright © 2005-2019, NumPy Developers.\newline
All rights reserved.

Redistribution and use in source and binary forms, with or without modification, are permitted provided that the following conditions are met:

\begin{itemize}
	\item Redistributions of source code must retain the above copyright notice, this list of conditions and the following disclaimer.
	\item Redistributions in binary form must reproduce the above copyright notice, this list of conditions and the following disclaimer in the documentation and/or other materials provided with the distribution.
	\item Neither the name of the NumPy Developers nor the names of any contributors may be used to endorse or promote products derived from this software without specific prior written permission.	
\end{itemize}

THIS SOFTWARE IS PROVIDED BY THE COPYRIGHT HOLDERS AND CONTRIBUTORS “AS IS” AND ANY EXPRESS OR IMPLIED WARRANTIES, INCLUDING, BUT NOT LIMITED TO, THE IMPLIED WARRANTIES OF MERCHANTABILITY AND FITNESS FOR A PARTICULAR PURPOSE ARE DISCLAIMED. IN NO EVENT SHALL THE COPYRIGHT OWNER OR CONTRIBUTORS BE LIABLE FOR ANY DIRECT, INDIRECT, INCIDENTAL, SPECIAL, EXEMPLARY, OR CONSEQUENTIAL DAMAGES (INCLUDING, BUT NOT LIMITED TO, PROCUREMENT OF SUBSTITUTE GOODS OR SERVICES; LOSS OF USE, DATA, OR PROFITS; OR BUSINESS INTERRUPTION) HOWEVER CAUSED AND ON ANY THEORY OF LIABILITY, WHETHER IN CONTRACT, STRICT LIABILITY, OR TORT (INCLUDING NEGLIGENCE OR OTHERWISE) ARISING IN ANY WAY OUT OF THE USE OF THIS SOFTWARE, EVEN IF ADVISED OF THE POSSIBILITY OF SUCH DAMAGE.

\textbf{PyInstaller}

================================\newline
The PyInstaller licensing terms\newline
================================

Copyright (c) 2010-2019, PyInstaller Development Team\newline
Copyright (c) 2005-2009, Giovanni Bajo\newline
Based on previous work under copyright (c) 2002 McMillan Enterprises, Inc.

PyInstaller is licensed under the terms of the GNU General Public License
as published by the Free Software Foundation; either version 2 of the License,
or (at your option) any later version.

Bootloader Exception\newline
--------------------

In addition to the permissions in the GNU General Public License, the
authors give you unlimited permission to link or embed compiled bootloader
and related files into combinations with other programs, and to distribute
those combinations without any restriction coming from the use of those
files. (The General Public License restrictions do apply in other respects;
for example, they cover modification of the files, and distribution when
not linked into a combine executable.)

Bootloader and Related Files\newline
----------------------------

Bootloader and related files are files which are embedded within the
final executable. This includes files in directories:

./bootloader/\newline
./PyInstaller/loader

About the PyInstaller Development Team\newline
--------------------------------------

The PyInstaller Development Team is the set of contributors
to the PyInstaller project. A full list with details is kept
in the documentation directory, in the file
``doc/CREDITS.rst``.

The core team that coordinates development on GitHub can be found here:\newline
https://github.com/pyinstaller/pyinstaller.  As of 2015, it consists of:

* Hartmut Goebel\newline
* Martin Zibricky\newline
* David Vierra\newline
* David Cortesi

Our Copyright Policy\newline
--------------------

PyInstaller uses a shared copyright model. Each contributor maintains copyright
over their contributions to PyInstaller. But, it is important to note that these
contributions are typically only changes to the repositories. Thus,
the PyInstaller source code, in its entirety is not the copyright of any single
person or institution.  Instead, it is the collective copyright of the entire
PyInstaller Development Team.  If individual contributors want to maintain
a record of what changes/contributions they have specific copyright on, they
should indicate their copyright in the commit message of the change, when they
commit the change to the PyInstaller repository.

With this in mind, the following banner should be used in any source code file
to indicate the copyright and license terms:

-----------------------------------------------------------------------------\newline
Copyright (c) 2005-20l5, PyInstaller Development Team.

Distributed under the terms of the GNU General Public License with exception
for distributing bootloader.

The full license is in the file COPYING.txt, distributed with this software.\newline
-----------------------------------------------------------------------------
\newpage
GNU General Public License\newline
--------------------------

https://gnu.org/licenses/gpl-2.0.html

\begin{center}
	GNU GENERAL PUBLIC LICENSE
	
	Version 2, June 1991
\end{center}

Copyright (C) 1989, 1991 Free Software Foundation, Inc.
51 Franklin Street, Fifth Floor, Boston, MA  02110-1301, USA
Everyone is permitted to copy and distribute verbatim copies
of this license document, but changing it is not allowed.

\begin{center}
	Preamble
\end{center}

The licenses for most software are designed to take away your
freedom to share and change it.  By contrast, the GNU General Public
License is intended to guarantee your freedom to share and change free
software--to make sure the software is free for all its users.  This
General Public License applies to most of the Free Software
Foundation's software and to any other program whose authors commit to
using it.  (Some other Free Software Foundation software is covered by
the GNU Library General Public License instead.)  You can apply it to
your programs, too.

When we speak of free software, we are referring to freedom, not
price.  Our General Public Licenses are designed to make sure that you
have the freedom to distribute copies of free software (and charge for
this service if you wish), that you receive source code or can get it
if you want it, that you can change the software or use pieces of it
in new free programs; and that you know you can do these things.

To protect your rights, we need to make restrictions that forbid
anyone to deny you these rights or to ask you to surrender the rights.
These restrictions translate to certain responsibilities for you if you
distribute copies of the software, or if you modify it.

For example, if you distribute copies of such a program, whether
gratis or for a fee, you must give the recipients all the rights that
you have.  You must make sure that they, too, receive or can get the
source code.  And you must show them these terms so they know their
rights.

We protect your rights with two steps: (1) copyright the software, and
(2) offer you this license which gives you legal permission to copy,
distribute and/or modify the software.

Also, for each author's protection and ours, we want to make certain
that everyone understands that there is no warranty for this free
software.  If the software is modified by someone else and passed on, we
want its recipients to know that what they have is not the original, so
that any problems introduced by others will not reflect on the original
authors' reputations.

Finally, any free program is threatened constantly by software
patents.  We wish to avoid the danger that redistributors of a free
program will individually obtain patent licenses, in effect making the
program proprietary.  To prevent this, we have made it clear that any
patent must be licensed for everyone's free use or not licensed at all.

The precise terms and conditions for copying, distribution and
modification follow.

GNU GENERAL PUBLIC LICENSE
TERMS AND CONDITIONS FOR COPYING, DISTRIBUTION AND MODIFICATION

0. This License applies to any program or other work which contains
a notice placed by the copyright holder saying it may be distributed
under the terms of this General Public License.  The "Program", below,
refers to any such program or work, and a "work based on the Program"
means either the Program or any derivative work under copyright law:
that is to say, a work containing the Program or a portion of it,
either verbatim or with modifications and/or translated into another
language.  (Hereinafter, translation is included without limitation in
the term "modification".)  Each licensee is addressed as "you".

Activities other than copying, distribution and modification are not
covered by this License; they are outside its scope.  The act of
running the Program is not restricted, and the output from the Program
is covered only if its contents constitute a work based on the
Program (independent of having been made by running the Program).
Whether that is true depends on what the Program does.

1. You may copy and distribute verbatim copies of the Program's
source code as you receive it, in any medium, provided that you
conspicuously and appropriately publish on each copy an appropriate
copyright notice and disclaimer of warranty; keep intact all the
notices that refer to this License and to the absence of any warranty;
and give any other recipients of the Program a copy of this License
along with the Program.

You may charge a fee for the physical act of transferring a copy, and
you may at your option offer warranty protection in exchange for a fee.

2. You may modify your copy or copies of the Program or any portion
of it, thus forming a work based on the Program, and copy and
distribute such modifications or work under the terms of Section 1
above, provided that you also meet all of these conditions:

a) You must cause the modified files to carry prominent notices
stating that you changed the files and the date of any change.

b) You must cause any work that you distribute or publish, that in
whole or in part contains or is derived from the Program or any
part thereof, to be licensed as a whole at no charge to all third
parties under the terms of this License.

c) If the modified program normally reads commands interactively
when run, you must cause it, when started running for such
interactive use in the most ordinary way, to print or display an
announcement including an appropriate copyright notice and a
notice that there is no warranty (or else, saying that you provide
a warranty) and that users may redistribute the program under
these conditions, and telling the user how to view a copy of this
License.  (Exception: if the Program itself is interactive but
does not normally print such an announcement, your work based on
the Program is not required to print an announcement.)

These requirements apply to the modified work as a whole.  If
identifiable sections of that work are not derived from the Program,
and can be reasonably considered independent and separate works in
themselves, then this License, and its terms, do not apply to those
sections when you distribute them as separate works.  But when you
distribute the same sections as part of a whole which is a work based
on the Program, the distribution of the whole must be on the terms of
this License, whose permissions for other licensees extend to the
entire whole, and thus to each and every part regardless of who wrote it.

Thus, it is not the intent of this section to claim rights or contest
your rights to work written entirely by you; rather, the intent is to
exercise the right to control the distribution of derivative or
collective works based on the Program.

In addition, mere aggregation of another work not based on the Program
with the Program (or with a work based on the Program) on a volume of
a storage or distribution medium does not bring the other work under
the scope of this License.

3. You may copy and distribute the Program (or a work based on it,
under Section 2) in object code or executable form under the terms of
Sections 1 and 2 above provided that you also do one of the following:

a) Accompany it with the complete corresponding machine-readable
source code, which must be distributed under the terms of Sections
1 and 2 above on a medium customarily used for software interchange; or,

b) Accompany it with a written offer, valid for at least three
years, to give any third party, for a charge no more than your
cost of physically performing source distribution, a complete
machine-readable copy of the corresponding source code, to be
distributed under the terms of Sections 1 and 2 above on a medium
customarily used for software interchange; or,

c) Accompany it with the information you received as to the offer
to distribute corresponding source code.  (This alternative is
allowed only for noncommercial distribution and only if you
received the program in object code or executable form with such
an offer, in accord with Subsection b above.)

The source code for a work means the preferred form of the work for
making modifications to it.  For an executable work, complete source
code means all the source code for all modules it contains, plus any
associated interface definition files, plus the scripts used to
control compilation and installation of the executable.  However, as a
special exception, the source code distributed need not include
anything that is normally distributed (in either source or binary
form) with the major components (compiler, kernel, and so on) of the
operating system on which the executable runs, unless that component
itself accompanies the executable.

If distribution of executable or object code is made by offering
access to copy from a designated place, then offering equivalent
access to copy the source code from the same place counts as
distribution of the source code, even though third parties are not
compelled to copy the source along with the object code.

4. You may not copy, modify, sublicense, or distribute the Program
except as expressly provided under this License.  Any attempt
otherwise to copy, modify, sublicense or distribute the Program is
void, and will automatically terminate your rights under this License.
However, parties who have received copies, or rights, from you under
this License will not have their licenses terminated so long as such
parties remain in full compliance.

5. You are not required to accept this License, since you have not
signed it.  However, nothing else grants you permission to modify or
distribute the Program or its derivative works.  These actions are
prohibited by law if you do not accept this License.  Therefore, by
modifying or distributing the Program (or any work based on the
Program), you indicate your acceptance of this License to do so, and
all its terms and conditions for copying, distributing or modifying
the Program or works based on it.

6. Each time you redistribute the Program (or any work based on the
Program), the recipient automatically receives a license from the
original licensor to copy, distribute or modify the Program subject to
these terms and conditions.  You may not impose any further
restrictions on the recipients' exercise of the rights granted herein.
You are not responsible for enforcing compliance by third parties to
this License.

7. If, as a consequence of a court judgment or allegation of patent
infringement or for any other reason (not limited to patent issues),
conditions are imposed on you (whether by court order, agreement or
otherwise) that contradict the conditions of this License, they do not
excuse you from the conditions of this License.  If you cannot
distribute so as to satisfy simultaneously your obligations under this
License and any other pertinent obligations, then as a consequence you
may not distribute the Program at all.  For example, if a patent
license would not permit royalty-free redistribution of the Program by
all those who receive copies directly or indirectly through you, then
the only way you could satisfy both it and this License would be to
refrain entirely from distribution of the Program.

If any portion of this section is held invalid or unenforceable under
any particular circumstance, the balance of the section is intended to
apply and the section as a whole is intended to apply in other
circumstances.

It is not the purpose of this section to induce you to infringe any
patents or other property right claims or to contest validity of any
such claims; this section has the sole purpose of protecting the
integrity of the free software distribution system, which is
implemented by public license practices.  Many people have made
generous contributions to the wide range of software distributed
through that system in reliance on consistent application of that
system; it is up to the author/donor to decide if he or she is willing
to distribute software through any other system and a licensee cannot
impose that choice.

This section is intended to make thoroughly clear what is believed to
be a consequence of the rest of this License.

8. If the distribution and/or use of the Program is restricted in
certain countries either by patents or by copyrighted interfaces, the
original copyright holder who places the Program under this License
may add an explicit geographical distribution limitation excluding
those countries, so that distribution is permitted only in or among
countries not thus excluded.  In such case, this License incorporates
the limitation as if written in the body of this License.

9. The Free Software Foundation may publish revised and/or new versions
of the General Public License from time to time.  Such new versions will
be similar in spirit to the present version, but may differ in detail to
address new problems or concerns.

Each version is given a distinguishing version number.  If the Program
specifies a version number of this License which applies to it and "any
later version", you have the option of following the terms and conditions
either of that version or of any later version published by the Free
Software Foundation.  If the Program does not specify a version number of
this License, you may choose any version ever published by the Free Software
Foundation.

10. If you wish to incorporate parts of the Program into other free
programs whose distribution conditions are different, write to the author
to ask for permission.  For software which is copyrighted by the Free
Software Foundation, write to the Free Software Foundation; we sometimes
make exceptions for this.  Our decision will be guided by the two goals
of preserving the free status of all derivatives of our free software and
of promoting the sharing and reuse of software generally.

NO WARRANTY

11. BECAUSE THE PROGRAM IS LICENSED FREE OF CHARGE, THERE IS NO WARRANTY
FOR THE PROGRAM, TO THE EXTENT PERMITTED BY APPLICABLE LAW.  EXCEPT WHEN
OTHERWISE STATED IN WRITING THE COPYRIGHT HOLDERS AND/OR OTHER PARTIES
PROVIDE THE PROGRAM "AS IS" WITHOUT WARRANTY OF ANY KIND, EITHER EXPRESSED
OR IMPLIED, INCLUDING, BUT NOT LIMITED TO, THE IMPLIED WARRANTIES OF
MERCHANTABILITY AND FITNESS FOR A PARTICULAR PURPOSE.  THE ENTIRE RISK AS
TO THE QUALITY AND PERFORMANCE OF THE PROGRAM IS WITH YOU.  SHOULD THE
PROGRAM PROVE DEFECTIVE, YOU ASSUME THE COST OF ALL NECESSARY SERVICING,
REPAIR OR CORRECTION.

12. IN NO EVENT UNLESS REQUIRED BY APPLICABLE LAW OR AGREED TO IN WRITING
WILL ANY COPYRIGHT HOLDER, OR ANY OTHER PARTY WHO MAY MODIFY AND/OR
REDISTRIBUTE THE PROGRAM AS PERMITTED ABOVE, BE LIABLE TO YOU FOR DAMAGES,
INCLUDING ANY GENERAL, SPECIAL, INCIDENTAL OR CONSEQUENTIAL DAMAGES ARISING
OUT OF THE USE OR INABILITY TO USE THE PROGRAM (INCLUDING BUT NOT LIMITED
TO LOSS OF DATA OR DATA BEING RENDERED INACCURATE OR LOSSES SUSTAINED BY
YOU OR THIRD PARTIES OR A FAILURE OF THE PROGRAM TO OPERATE WITH ANY OTHER
PROGRAMS), EVEN IF SUCH HOLDER OR OTHER PARTY HAS BEEN ADVISED OF THE
POSSIBILITY OF SUCH DAMAGES.

END OF TERMS AND CONDITIONS

\textbf{Python}

1. This LICENSE AGREEMENT is between the Python Software Foundation ("PSF"), and
the Individual or Organization ("Licensee") accessing and otherwise using Python
3.7.3 software in source or binary form and its associated documentation.

2. Subject to the terms and conditions of this License Agreement, PSF hereby
grants Licensee a nonexclusive, royalty-free, world-wide license to reproduce,
analyze, test, perform and/or display publicly, prepare derivative works,
distribute, and otherwise use Python 3.7.3 alone or in any derivative
version, provided, however, that PSF's License Agreement and PSF's notice of
copyright, i.e., "Copyright © 2001-2019 Python Software Foundation; All Rights
Reserved" are retained in Python 3.7.3 alone or in any derivative version
prepared by Licensee.

3. In the event Licensee prepares a derivative work that is based on or
incorporates Python 3.7.3 or any part thereof, and wants to make the
derivative work available to others as provided herein, then Licensee hereby
agrees to include in any such work a brief summary of the changes made to Python
3.7.3.

4. PSF is making Python 3.7.3 available to Licensee on an "AS IS" basis.
PSF MAKES NO REPRESENTATIONS OR WARRANTIES, EXPRESS OR IMPLIED.  BY WAY OF
EXAMPLE, BUT NOT LIMITATION, PSF MAKES NO AND DISCLAIMS ANY REPRESENTATION OR
WARRANTY OF MERCHANTABILITY OR FITNESS FOR ANY PARTICULAR PURPOSE OR THAT THE
USE OF PYTHON 3.7.3 WILL NOT INFRINGE ANY THIRD PARTY RIGHTS.

5. PSF SHALL NOT BE LIABLE TO LICENSEE OR ANY OTHER USERS OF PYTHON 3.7.3
FOR ANY INCIDENTAL, SPECIAL, OR CONSEQUENTIAL DAMAGES OR LOSS AS A RESULT OF
MODIFYING, DISTRIBUTING, OR OTHERWISE USING PYTHON 3.7.3, OR ANY DERIVATIVE
THEREOF, EVEN IF ADVISED OF THE POSSIBILITY THEREOF.

6. This License Agreement will automatically terminate upon a material breach of
its terms and conditions.

7. Nothing in this License Agreement shall be deemed to create any relationship
of agency, partnership, or joint venture between PSF and Licensee.  This License
Agreement does not grant permission to use PSF trademarks or trade name in a
trademark sense to endorse or promote products or services of Licensee, or any
third party.

8. By copying, installing or otherwise using Python 3.7.3, Licensee agrees
to be bound by the terms and conditions of this License Agreement.

\textbf{Requests}

Copyright 2018 Kenneth Reitz

Licensed under the Apache License, Version 2.0 (the “License”); you may not use this file except in compliance with the License. You may obtain a copy of the License \href{https://www.apache.org/licenses/LICENSE-2.0}{here}.

Unless required by applicable law or agreed to in writing, software distributed under the License is distributed on an “AS IS” BASIS, WITHOUT WARRANTIES OR CONDITIONS OF ANY KIND, either express or implied. See the License for the specific language governing permissions and limitations under the License.

\textbf{wxPython}

\textit{Preamble}

The licencing of the wxWidgets library is intended to protect the wxWidgets
library, its developers, and its users, so that the considerable investment it
represents is not abused.

Under the terms of the original wxWidgets licences, you as a user are not
obliged to distribute wxWidgets source code with your products, if you
distribute these products in binary form. However, you are prevented from
restricting use of the library in source code form, or denying others the
rights to use or distribute wxWidgets library source code in the way intended.

The wxWindows Library License establishes the copyright for the code and
related material, and it gives you legal permission to copy, distribute and/or
modify the library. It also asserts that no warranty is given by the authors
for this or derived code.

The core distribution of the wxWidgets library contains files under two
different licences:

* Most files are distributed under the GNU Library General Public License,
version 2, with the special exception that you may create and distribute
object code versions built from the source code or modified versions of it
(even if these modified versions include code under a different licence),
and distribute such binaries under your own terms.

* Most core wxWidgets manuals are made available under the "wxWindows Free
Documentation License", which allows you to distribute modified versions of
the manuals, such as versions documenting any modifications made by you in
your version of the library. However, you may not restrict any third party
from reincorporating your changes into the original manuals.

\textit{wxWindows Library Licence}

\begin{center}
	wxWindows Library Licence, Version 3.1
	======================================
\end{center}

Copyright (c) 1998-2005 Julian Smart, Robert Roebling et al

Everyone is permitted to copy and distribute verbatim copies
of this licence document, but changing it is not allowed.

\begin{center}
	WXWINDOWS LIBRARY LICENCE
	TERMS AND CONDITIONS FOR COPYING, DISTRIBUTION AND MODIFICATION
\end{center}

This library is free software; you can redistribute it and/or modify it
under the terms of the GNU Library General Public Licence as published by
the Free Software Foundation; either version 2 of the Licence, or (at your
option) any later version.

This library is distributed in the hope that it will be useful, but WITHOUT
ANY WARRANTY; without even the implied warranty of MERCHANTABILITY or
FITNESS FOR A PARTICULAR PURPOSE.  See the GNU Library General Public
Licence for more details.

You should have received a copy of the GNU Library General Public Licence
along with this software, usually in a file named COPYING.LIB.  If not,
write to the Free Software Foundation, Inc., 51 Franklin Street, Fifth
Floor, Boston, MA 02110-1301 USA.

EXCEPTION NOTICE

1. As a special exception, the copyright holders of this library give
permission for additional uses of the text contained in this release of the
library as licenced under the wxWindows Library Licence, applying either
version 3.1 of the Licence, or (at your option) any later version of the
Licence as published by the copyright holders of version 3.1 of the Licence
document.

2. The exception is that you may use, copy, link, modify and distribute
under your own terms, binary object code versions of works based on the
Library.

3. If you copy code from files distributed under the terms of the GNU
General Public Licence or the GNU Library General Public Licence into a
copy of this library, as this licence permits, the exception does not apply
to the code that you add in this way.  To avoid misleading anyone as to the
status of such modified files, you must delete this exception notice from
such code and/or adjust the licensing conditions notice accordingly.

4. If you write modifications of your own for this library, it is your
choice whether to permit this exception to apply to your modifications.  If
you do not wish that, you must delete the exception notice from such code
and/or adjust the licensing conditions notice accordingly.

\textit{wxWindows Free Documentation License}

\begin{center}
	wxWindows Free Documentation Licence, Version 3
	===============================================
\end{center}

Copyright (c) 1998 Julian Smart, Robert Roebling et al

Everyone is permitted to copy and distribute verbatim copies
of this licence document, but changing it is not allowed.

\begin{center}
	WXWINDOWS FREE DOCUMENTATION LICENCE
	TERMS AND CONDITIONS FOR COPYING, DISTRIBUTION AND MODIFICATION
\end{center}

1. Permission is granted to make and distribute verbatim copies of this
manual or piece of documentation provided any copyright notice and this
permission notice are preserved on all copies.

2. Permission is granted to process this file or document through a
document processing system and, at your option and the option of any third
party, print the results, provided a printed document carries a copying
permission notice identical to this one.

3. Permission is granted to copy and distribute modified versions of this
manual or piece of documentation under the conditions for verbatim copying,
provided also that any sections describing licensing conditions for this
manual, such as, in particular, the GNU General Public Licence, the GNU
Library General Public Licence, and any wxWindows Licence are included
exactly as in the original, and provided that the entire resulting derived
work is distributed under the terms of a permission notice identical to
this one.

4. Permission is granted to copy and distribute translations of this manual
or piece of documentation into another language, under the above conditions
for modified versions, except that sections related to licensing, including
this paragraph, may also be included in translations approved by the
copyright holders of the respective licence documents in addition to the
original English.

\begin{center}
	WARRANTY DISCLAIMER
\end{center}

5. BECAUSE THIS MANUAL OR PIECE OF DOCUMENTATION IS LICENSED FREE OF
CHARGE, THERE IS NO WARRANTY FOR IT, TO THE EXTENT PERMITTED BY APPLICABLE
LAW.  EXCEPT WHEN OTHERWISE STATED IN WRITING THE COPYRIGHT HOLDERS AND/OR
OTHER PARTIES PROVIDE THIS MANUAL OR PIECE OF DOCUMENTATION "AS IS" WITHOUT
WARRANTY OF ANY KIND, EITHER EXPRESSED OR IMPLIED, INCLUDING, BUT NOT
LIMITED TO, THE IMPLIED WARRANTIES OF MERCHANTABILITY AND FITNESS FOR A
PARTICULAR PURPOSE.  THE ENTIRE RISK AS TO THE QUALITY AND PERFORMANCE OF
THE MANUAL OR PIECE OF DOCUMENTATION IS WITH YOU.  SHOULD THE MANUAL OR
PIECE OF DOCUMENTATION PROVE DEFECTIVE, YOU ASSUME THE COST OF ALL
NECESSARY SERVICING, REPAIR OR CORRECTION.

6. IN NO EVENT UNLESS REQUIRED BY APPLICABLE LAW OR AGREED TO IN WRITING
WILL ANY COPYRIGHT HOLDER, OR ANY OTHER PARTY WHO MAY MODIFY AND/OR
REDISTRIBUTE THE MANUAL OR PIECE OF DOCUMENTATION AS PERMITTED ABOVE, BE
LIABLE TO YOU FOR DAMAGES, INCLUDING ANY GENERAL, SPECIAL, INCIDENTAL OR
CONSEQUENTIAL DAMAGES ARISING OUT OF THE USE OR INABILITY TO USE THE MANUAL
OR PIECE OF DOCUMENTATION (INCLUDING BUT NOT LIMITED TO LOSS OF DATA OR
DATA BEING RENDERED INACCURATE OR LOSSES SUSTAINED BY YOU OR THIRD PARTIES
OR A FAILURE OF A PROGRAM BASED ON THE MANUAL OR PIECE OF DOCUMENTATION TO
OPERATE WITH ANY OTHER PROGRAMS), EVEN IF SUCH HOLDER OR OTHER PARTY HAS
BEEN ADVISED OF THE POSSIBILITY OF SUCH DAMAGES.
