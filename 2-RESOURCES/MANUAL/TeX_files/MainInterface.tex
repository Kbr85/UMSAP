\chapter{Work flow in Utilities for Mass Spectrometry Analysis of Proteins}
\label{chap:workflow}

When you start UMSAP, the program will display the main window (\autoref{fig:mainwindow}). From this window you can access all the modules and utilities either by the menu entries: Modules and Utilities or by the corresponding buttons in the right side list. A complete description of each module and utility is given in the following chapters.

\begin{figure}[h]
	\centering
	\includegraphics[width=0.7\textwidth]{./IMAGES/MAIN-WINDOW/20190425-mainwindow.jpg}	    
	\caption[The main window of UMSAP]{\textbf{The main window of UMSAP.} From this window users can access all the available Modules and Utilities.} 
	\label{fig:mainwindow}
	\vspace{-5pt} 	
\end{figure}  

\section{The data files}
\label{sec:datafile}

The main data file for UMSAP is the file containing the detected peptide sequences after all peak assignments have been completed. The program expects this file to be a plain text file containing a table with the data. Columns in the file are expected to be tab separated. The first row in the file is expected to contain only the names of the columns. There is no limitation in the amount and type of data present in the data file. However, each module will expect certain columns to be present. Columns not needed by the modules will simply be ignored.

The module Targeted Proteolysis requires a second data file containing the sequence of the recombinant protein used in the experiments. The sequence file must be a FASTA formatted file or a plain text file. Either way, only one sequence is expected to be present in the file. In the case of a plain text file only the sequence of the recombinant protein using one letter code for amino acids must be present in the file. In addition, a second sequence file with the sequence of the native target protein and a pdb file might be specify in the Targeted Proteolysis module. Alternatively, a UNIPROT or PDB code can be given for the sequences and pdb files, respectively.

Certain output files generated by UMSAP can be used as data file as well. Files that can be used this way will contain the results from previous data post-processing. These output files will be plain text files, since in some cases users will need to directly read the data contained in these files. However, it is important that these files remain unchanged to ensure the correct display of the results contained in the files by UMSAP.

\section{Using Utilities for Mass Spectrometry Analysis of Proteins}

Once you have your data files, using UMSAP is straightforward. Just open the program and select a module or utility. In the new window, fill in the needed information and hit the Start button in the bottom right corner. Depending on the amount of data and the complexity of the analysis to perform it may take a few minutes for the program to complete the task at hand. In each window performing an analysis, there is a progress bar near the Start button that gives a rough guess of the remaining time needed to complete the current analysis.

Currently, the text boxes showing the path to a file  do not have auto-complete capabilities. In addition, relative paths are not supported. Therefore, the full paths to files are expected. In the case of output files and folders if the text boxes are left empty, appropriate defaults values will be provided.  

In order to make the program as user friendly as possible help messages will pop up from buttons and labels. The help messages will contain a brief description of what is the button or label for and what input is expected from the user. In this way, users can find basic information about a particular element of the interface without needing to go to the manual or online tutorials. If more information is needed, users may consult the manual or click the Help button in the bottom left corner of the module/utility window to read an online tutorial. 

Depending on the module or utility just run, new windows will be created to show a graphical representation of the results.

Special care have been taken to handle errors that may appear during the processing of the results. Errors will be reported to users using dialog boxes with plain English messages so users may correct the errors right away and continue with the analysis. In the case of an unforeseen error occurring, the message error will be more code oriented. It will be helpful if users send a crash report to umsap-crashreport@umsap.nl so we can correct the error and/or create an appropriate error message. 

In general, windows showing a graphical representation of results will be allowed to resize while module/utilities windows will not. In addition, windows showing a graphical representation of results will have a Tools menu entry with commands that are specific for each window, for example, to save an image of the displayed results. In most of these windows the functionalities in the Tools menu can be also accessed with the right mouse button.

For the modules, UMSAP will create a .uscr file after finishing the analysis. This is an input file that can be used to run an analysis one more time without users having to type in again all the needed information. Once the input file is selected, using the Run Input File entry in the Script menu, the appropriate module will be launched and the information found in the .uscr file will be used to fill the needed data in the interface of the module. At this point, users may decide to update the analysis by changing some of the parameters in the interface of the module. Clicking the Start button will start the analysis of the data files.   

\section{Navigating through Utilities for Mass Spectrometry Analysis of Proteins}

The entries Modules and Utilities will be available in the menu of every window. The Modules entry in the menu gives direct access to all modules. The same is true for the Utilities entry. These menu entries are the fastest way to access all the functions in UMSAP. In a typical UMSAP session, users will work with different independent windows simultaneously. The windows have descriptive names so users can quickly guess the content of any window. The scheme of the windows name is \textit{UMSAP - Utilities or Module Name - Name of the window}. For example, the window with name \textit{UMSAP - Utilities - Correlation coefficients (t\num[detect-all]{1}.corr)} will be displaying the correlation coefficient matrix saved in the file t\num{1}.corr.

\section{The output files}

In order to minimize user input, the names for output files and output folders can be left unspecified. In this case, UMSAP will use default names for files and folders. When names for files and folders are given, it is recommended to use names with no spaces, e.g. my-output-file or myoutputfile.The software will try to avoid to overwrite previous output files or folders. If a selected folder already exist and it is not empty, UMSAP will create a new folder to write in the results. The location of the new folder will be the same as the selected folder. The name of the new folder will be the same as the selected folder plus the current date and time to the second, for example, selectedfolder-20190425092904. The same is true for files unless the user forces UMSAP to overwrite a file. 

\section{Backward compatibility}
\label{sec:backcomp}

UMSAP is capable to read the .tarprot file from previous versions. Other files generated by UMSAP are version dependent and cannot be read with different versions of UMSAP. Nevertheless, two tools allows to quickly generate any UMSAP file based on a .tarprot file from a previous version of UMSAP. These tools are described in \autoref{subsec:updateres} and \autoref{subsec:reanalyzetarprot}. How to generate a .tarprot file is explained in \autoref{chap:enzdig}.